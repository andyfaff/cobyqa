%% Generated by Sphinx.
\def\sphinxdocclass{report}
\documentclass[letterpaper,10pt,english]{sphinxmanual}
\ifdefined\pdfpxdimen
   \let\sphinxpxdimen\pdfpxdimen\else\newdimen\sphinxpxdimen
\fi \sphinxpxdimen=.75bp\relax
\ifdefined\pdfimageresolution
    \pdfimageresolution= \numexpr \dimexpr1in\relax/\sphinxpxdimen\relax
\fi
%% let collapsible pdf bookmarks panel have high depth per default
\PassOptionsToPackage{bookmarksdepth=5}{hyperref}
%% turn off hyperref patch of \index as sphinx.xdy xindy module takes care of
%% suitable \hyperpage mark-up, working around hyperref-xindy incompatibility
\PassOptionsToPackage{hyperindex=false}{hyperref}
%% memoir class requires extra handling
\makeatletter\@ifclassloaded{memoir}
{\ifdefined\memhyperindexfalse\memhyperindexfalse\fi}{}\makeatother

\PassOptionsToPackage{warn}{textcomp}

\catcode`^^^^00a0\active\protected\def^^^^00a0{\leavevmode\nobreak\ }
\usepackage{cmap}
\usepackage[LGR,T1]{fontenc}
\usepackage{amsmath,amssymb,amstext}
\usepackage{polyglossia}
\setmainlanguage{english}



\setmainfont{FreeSerif}[
  Extension      = .otf,
  UprightFont    = *,
  ItalicFont     = *Italic,
  BoldFont       = *Bold,
  BoldItalicFont = *BoldItalic
]
\setsansfont{FreeSans}[
  Extension      = .otf,
  UprightFont    = *,
  ItalicFont     = *Oblique,
  BoldFont       = *Bold,
  BoldItalicFont = *BoldOblique,
]
\setmonofont{FreeMono}[
  Extension      = .otf,
  UprightFont    = *,
  ItalicFont     = *Oblique,
  BoldFont       = *Bold,
  BoldItalicFont = *BoldOblique,
]



\usepackage[Bjarne]{fncychap}
\usepackage{sphinx}

\fvset{fontsize=\small}
\usepackage{geometry}


% Include hyperref last.
\usepackage{hyperref}
% Fix anchor placement for figures with captions.
\usepackage{hypcap}% it must be loaded after hyperref.
% Set up styles of URL: it should be placed after hyperref.
\urlstyle{same}


\usepackage{sphinxmessages}
\setcounter{tocdepth}{1}


\usepackage{dsfont}

% Constants and mathematical functions in Roman style font.
\def\eu{\ensuremath{\mathrm{e}}}
\def\iu{\ensuremath{\mathrm{i}}}
\def\du{\ensuremath{\mathrm{d}}}
\DeclareMathOperator{\vspan}{span}

% Extra mathematical functions
\newcommand{\abs}[2][]{#1\lvert#2#1\rvert}
\newcommand{\ceil}[2][]{#1\lceil#2#1\rceil}
\newcommand{\floor}[2][]{#1\lfloor#2#1\rfloor}
\newcommand{\norm}[2][]{#1\lVert#2#1\rVert}
\newcommand{\set}[2][]{#1\{#2#1\}}
\newcommand{\inner}[2][]{#1\langle#2#1\rangle}

% Sets in blackboard-bold style font.
\def\C{\ensuremath{\mathds{C}}}
\def\N{\ensuremath{\mathds{N}}}
\def\Q{\ensuremath{\mathds{Q}}}
\def\R{\ensuremath{\mathds{R}}}
\def\Z{\ensuremath{\mathds{Z}}}

% Mathematical operators in sans serif style font
\def\T{\ensuremath{\mathsf{T}}}
    

\title{COBYQA Reference}
\date{October 26, 2021}
\release{1.0.dev0}
\author{Tom M. Ragonneau}
\newcommand{\sphinxlogo}{\vbox{}}
\renewcommand{\releasename}{Release}
\makeindex
\begin{document}

\pagestyle{empty}
\sphinxmaketitle
\pagestyle{plain}
\sphinxtableofcontents
\pagestyle{normal}
\phantomsection\label{\detokenize{algo/index::doc}}

\begin{quote}\begin{description}
\item[{Date}] \leavevmode
\sphinxAtStartPar
October 26, 2021

\end{description}\end{quote}

\sphinxAtStartPar
This section presents in depth the derivative trust\sphinxhyphen{}region method employed by
COBYQA to tackle nonlinear constrained problems.


\chapter{Optimization framework (\sphinxstyleliteralintitle{\sphinxupquote{cobyqa}})}
\label{\detokenize{algo/optimize:optimization-framework-cobyqa}}\label{\detokenize{algo/optimize:optimize}}\label{\detokenize{algo/optimize::doc}}

\sphinxstrong{See also:}
\nopagebreak


\sphinxAtStartPar
\sphinxcode{\sphinxupquote{minimize}},
\sphinxcode{\sphinxupquote{OptimizeResult}}




\section{Statement of the problem}
\label{\detokenize{algo/optimize:statement-of-the-problem}}

\chapter{Linear algebra (\sphinxstyleliteralintitle{\sphinxupquote{cobyqa.linalg}})}
\label{\detokenize{algo/linalg:linear-algebra-cobyqa-linalg}}\label{\detokenize{algo/linalg:linalg}}\label{\detokenize{algo/linalg::doc}}
\sphinxAtStartPar
This section presents the mathematical frameworks underneath the subproblem
solvers of COBYQA.


\sphinxstrong{See also:}
\nopagebreak


\sphinxAtStartPar
\sphinxcode{\sphinxupquote{bvcs}},
\sphinxcode{\sphinxupquote{bvlag}},
\sphinxcode{\sphinxupquote{bvtcg}},
\sphinxcode{\sphinxupquote{cpqp}},
\sphinxcode{\sphinxupquote{givens}},
\sphinxcode{\sphinxupquote{lctcg}},
\sphinxcode{\sphinxupquote{nnls}},
\sphinxcode{\sphinxupquote{qr}}




\section{Geometry of the interpolation set}
\label{\detokenize{algo/linalg.altmov:geometry-of-the-interpolation-set}}\label{\detokenize{algo/linalg.altmov:linalg-altmov}}\label{\detokenize{algo/linalg.altmov::doc}}

\section{Bound constrained truncated conjugate gradient}
\label{\detokenize{algo/linalg.bvtcg:bound-constrained-truncated-conjugate-gradient}}\label{\detokenize{algo/linalg.bvtcg:linalg-bvtcg}}\label{\detokenize{algo/linalg.bvtcg::doc}}
\sphinxAtStartPar
When no linear nor nonlinear constraints are provided to COBYQA (that is when
the considered problem is bound constrained), the trust\sphinxhyphen{}region subproblem to
solve at each iteration is of the form
\begin{equation}\label{equation:algo/linalg.bvtcg:bvtcg}
\begin{split}\begin{array}{ll}
    \min        & \quad f(x) = \inner{x, g} + \frac{1}{2} \inner{x, H x}\\
    \text{s.t.} & \quad l \le x \le u,\\
                & \quad \norm{x} \le \Delta,\\
                & \quad x \in \R^n,
\end{array}\end{split}
\end{equation}
\sphinxAtStartPar
where \(g \in \R^n\) approximates the gradient of the nonlinear objective
function at the origin, \(H \in \R^{n \times n}\) is a symmetric matrix
that approximates the Hessian matrix of the nonlinear objective function at the
origin, \(l \in \R^n\) and \(u \in \R^n\) are the lower and upper
bounds of the problems (with \(l < u\)), \(\Delta > 0\) is the current
trust\sphinxhyphen{}region radius, and \(\norm{\cdot}\) is the Euclidean norm.


\subsection{The unconstrained case}
\label{\detokenize{algo/linalg.bvtcg:the-unconstrained-case}}\label{\detokenize{algo/linalg.bvtcg:tcg-base}}
\sphinxAtStartPar
We assume in this section that the lower and upper bounds in \eqref{equation:algo/linalg.bvtcg:bvtcg} are
respectively set to \(-\infty\) and \(+\infty\). Powell showed in
\sphinxcite{algo/linalg.bvtcg:powe75} that a trust\sphinxhyphen{}region method is convergent if the trust\sphinxhyphen{}region step
\(x^{\ast} \in \R^n\) satisfies
\begin{equation*}
\begin{split}f(x^0) - f(x^{\ast}) \ge \gamma \norm{g} \min \set{\Delta, \norm{g} / \norm{H}},\end{split}
\end{equation*}
\sphinxAtStartPar
for some \(\gamma > 0\), where \(\norm{\cdot}\) is the Euclidean norm.
It is easy to see that a Cauchy step satisfies such a condition
(with \(\gamma = 1/2\)). Therefore, to preserve the computational
efficiency of a trust\sphinxhyphen{}region method, it is usual to solve inexactly problem
\eqref{equation:algo/linalg.bvtcg:bvtcg} using the truncated conjugate gradient method of Steihaug \sphinxcite{algo/linalg.lctcg:stei83}
and Toint \sphinxcite{algo/linalg.lctcg:toin81}. Given the initial values \(x^0 = 0\) and
\(d^0 = -g\), it generates the sequence of iterates
\begin{equation*}
\begin{split}\left\{
\begin{array}{l}
    \alpha_k = -\inner{d^k, g^k} / \inner{d^k, Hd^k},\\
    \beta_k = \norm{g^{k + 1}}^2 / \norm{g^k}^2,\\
    x^{k + 1} = x^k + \alpha_k d^k,\\
    d^{k + 1} = -g^k + \beta_k d^k,
\end{array}
\right.\end{split}
\end{equation*}
\sphinxAtStartPar
where \(g^k = \nabla f(x^k) = g + Hx^k\). The computations are stopped if
either
\begin{enumerate}
\sphinxsetlistlabels{\arabic}{enumi}{enumii}{}{.}%
\item {} 
\sphinxAtStartPar
\(g^k = 0\) and \(\inner{d^k, Hd^k} \ge 0\), in which case the
global minimizer is found; or

\item {} 
\sphinxAtStartPar
\(\norm{x^{k + 1}} \ge \Delta\) or \(\inner{d^k, Hd^k} < 0\), in
which case \(x^k + \alpha_{\Delta} d^k\) is returned, where
\(\alpha_{\Delta} > 0\) is chosen so that
\(\norm{x^k + \alpha_{\Delta} d^k} = \Delta\).

\end{enumerate}

\sphinxAtStartPar
It is known that the truncated conjugate gradient method terminates after at
most \(n\) iterations, and that the reduction in the objective function
obtained with the truncated conjugate gradient method is at least half of the
reduction obtained by the global minimizer.


\subsection{The constrained case}
\label{\detokenize{algo/linalg.bvtcg:the-constrained-case}}
\sphinxAtStartPar
We assume in this section that at least some values of the lower and upper
bounds in \eqref{equation:algo/linalg.bvtcg:bvtcg} are finite. When no linear nor nonlinear constraints are
provided, COBYQA solves the trust\sphinxhyphen{}region subproblem using the TRSBOX algorithm
\sphinxcite{algo/linalg.bvtcg:powe09}, which is presented below.


\subsubsection{The bound constrained truncated conjugate gradient procedure}
\label{\detokenize{algo/linalg.bvtcg:the-bound-constrained-truncated-conjugate-gradient-procedure}}
\sphinxAtStartPar
The strategy employed by \sphinxcode{\sphinxupquote{bvtcg}} to tackle the bound constraints in the
truncated conjugate gradient procedure is the use of an active set. At each
iteration of the method, a truncate conjugate gradient step is performed on the
coordinates that are not fixed by the active set. If a new bound is hit during
such iteration, the bound is added to the active set, and the procedure is
restarted. The active set is only enlarged through the iterations, which then
ensures the termination of the method.

\sphinxAtStartPar
The initial active set is a subset of the active bounds at the origin. Clearly,
a active bound should not be included in the active set if a Cauchy step (a
positive step along \(-g\)) would depart from the bound, as the bound is
never removed from the active set. The complete framework of \sphinxcode{\sphinxupquote{bvtcg}} is
described below. For sake of clarity, we denote \(\mathcal{I}\) the active
set and \(\Pi(v)\) the vector whose \(i\)\sphinxhyphen{}th coordinate is \(v_i\)
if \(i \notin \mathcal{I}\), and zero otherwise.
\begin{enumerate}
\sphinxsetlistlabels{\arabic}{enumi}{enumii}{}{.}%
\item {} 
\sphinxAtStartPar
Set \(x^0 = 0\) and the active set \(\mathcal{I}\) to the indices
for which either \(l_i = 0\) and \(g_i \ge 0\) or \(u_i = 0\)
and \(g_i \le 0\).

\item {} 
\sphinxAtStartPar
Set \(d^0 = -\Pi(g)\), \(g^0 = g\), and \(k = 0\).

\item {} 
\sphinxAtStartPar
Let \(\alpha_{\Delta, k}\) be the largest number such that
\(\norm{x^k + \alpha_{\Delta, k} d^k} = \Delta\).

\item {} 
\sphinxAtStartPar
Let \(\alpha_{Q, k}\) be \(-\inner{d^k, g^k} / \inner{d^k, Hd^k}\)
if \(\inner{d^k, Hd^k} > 0\) and \(+\infty\) otherwise.

\item {} 
\sphinxAtStartPar
Let \(\alpha_{B, k}\) be the largest number such that
\(l \le x^k + \alpha_{B, k} d^k \le u\) and
\(\alpha_k = \min \set{\alpha_{\Delta, k}, \alpha_{Q, k}, \alpha_{B, k}}\).

\item {} 
\sphinxAtStartPar
Update \(x^{k + 1} = x^k + \alpha_k d^k\),
\(g^{k + 1} = g^k + \alpha_k H d^k\), and
\(\beta_k = \norm{g^{k + 1}}^2 / \norm{g^k}^2\).

\item {} 
\sphinxAtStartPar
If \(\alpha_k = \alpha_{\Delta, k}\) or \(g^{k + 1} = 0\), stop the
computations.

\item {} 
\sphinxAtStartPar
If \(\alpha_k = \alpha_{B, k}\), add a new active coordinate to
\(\mathcal{I}\), set \(x^0 = x^{k + 1}\), and go to step 2.

\item {} 
\sphinxAtStartPar
Update \(d^{k + 1} = -\Pi(g^k) + \beta_k d^k\), increment \(k\), and
go to step 3.

\end{enumerate}


\subsubsection{Further refinement of the trial step}
\label{\detokenize{algo/linalg.bvtcg:further-refinement-of-the-trial-step}}
\sphinxAtStartPar
If the step \(x^k\) returned by the constrained truncated conjugate
gradient procedure satisfies \(\norm{x^k} = \Delta\), it is likely that
the objective function in \eqref{equation:algo/linalg.bvtcg:bvtcg} can be further decreased by moving this
point round the trust\sphinxhyphen{}region boundary. In fact, the global solution of problem
\eqref{equation:algo/linalg.bvtcg:bvtcg} is on the trust\sphinxhyphen{}region boundary. The method \sphinxcode{\sphinxupquote{bvtcg}} then may
further reduce the function evaluation by returning in this case an approximate
solution to
\begin{equation*}
\begin{split}\begin{array}{ll}
    \min        & \quad f(x) = \inner{x - x^0, g} + \frac{1}{2} \inner{x - x^0, H (x - x^0)}\\
    \text{s.t.} & \quad l \le x \le u,\\
                & \quad \norm{x - x^0} = \Delta,\\
                & \quad x \in \vspan \set{\Pi(x^k), \Pi(g^k)} \subseteq \R^n.
\end{array}\end{split}
\end{equation*}
\sphinxAtStartPar
To do so, the method builds an orthogonal basis \(\set{\Pi(x^k), s}\) of
\(\vspan \set{\Pi(x^k), \Pi(g^k)}\) by selecting the vector
\(s \in \R^n\) such that \(\inner{s, \Pi(x^k)} = 0\),
\(\inner{s, \Pi(g^k)} < 0\), and \(\norm{s} = \norm{\Pi(x^k)}\).
Further, the method considers the function
\(x(\theta) = x^k + (\cos \theta - 1) \Pi(x^k) + \sin \theta s\) with
\(0 \le \theta \le \pi / 4\) and solves approximately
\begin{equation*}
\begin{split}\begin{array}{ll}
    \min        & \quad f(x(\theta))\\
    \text{s.t.} & \quad l \le x(\theta) \le u,\\
                & \quad 0 \le \theta \le \pi / 4,
\end{array}\end{split}
\end{equation*}
\sphinxAtStartPar
the trust\sphinxhyphen{}region condition being automatically ensured by the choice of
\(s\). If the value of \(\theta\) is restricted by a bound, it is added
to the active set \(\mathcal{I}\), and the refinement procedure is
restarted. Since the active set is very reduced, this procedure terminates in
at most \(n - \abs{\mathcal{I}}\), where \(\abs{\mathcal{I}}\) denotes
the number of active bounds at the end of the constrained truncated conjugate
gradient procedure.
\subsubsection*{References}


\section{Convex piecewise quadratic programming}
\label{\detokenize{algo/linalg.cpqp:convex-piecewise-quadratic-programming}}\label{\detokenize{algo/linalg.cpqp:linalg-cpqp}}\label{\detokenize{algo/linalg.cpqp::doc}}
\sphinxAtStartPar
In general (that is when some linear and/or nonlinear constraints are
provided), to determine a trust\sphinxhyphen{}region normal step, COBYQA must solve a problem
of the form
\begin{equation}\label{equation:algo/linalg.cpqp:cpqp}
\begin{split}\begin{array}{ll}
    \min        & \quad f(x) = \frac{1}{2} \big(\norm{[Ax - b]_+}^2 + \norm{Cx - d}^2\big)\\
    \text{s.t.} & \quad l \le x \le u,\\
                & \quad \norm{x} \le \Delta,\\
                & \quad x \in \R^n,
\end{array}\end{split}
\end{equation}
\sphinxAtStartPar
where \(A \in \R^{m_1 \times n}\), \(b \in \R^{m_1}\),
\(C \in \R^{m_2 \times n}\), \(d \in \R^{m_2}\), \(l \in \R^n\) and
\(u \in \R^n\) are the lower and upper bounds of the problems
(with \(l < u\)), \(\Delta > 0\) is the current trust\sphinxhyphen{}region radius,
and \(\norm{\cdot}\) is the Euclidean norm.


\section{Linear constrained truncated conjugate gradient}
\label{\detokenize{algo/linalg.lctcg:linear-constrained-truncated-conjugate-gradient}}\label{\detokenize{algo/linalg.lctcg:linalg-lctcg}}\label{\detokenize{algo/linalg.lctcg::doc}}
\sphinxAtStartPar
In general (that is when some linear and/or nonlinear constraints are
provided), to determine a trust\sphinxhyphen{}region tangential step, COBYQA must solve a
problem of the form
\begin{equation}\label{equation:algo/linalg.lctcg:lctcg}
\begin{split}\begin{array}{ll}
    \min        & \quad f(x) = \inner{x, g} + \frac{1}{2} \inner{x, H x}\\
    \text{s.t.} & \quad Ax \le b,\\
                & \quad Cx = d,\\
                & \quad \norm{x} \le \Delta,\\
                & \quad x \in \R^n,
\end{array}\end{split}
\end{equation}
\sphinxAtStartPar
where \(g \in \R^n\) approximates the gradient of the nonlinear objective
function at the origin, \(H \in \R^{n \times n}\) is a symmetric matrix
that approximates the Hessian matrix of the nonlinear objective function at the
origin, \(A \in \R^{m_1 \times n}\) and \(C \in \R^{m_2 \times n}\) are
the Jacobian matrices of the linear inequality and equality constraints,
\(b \in \R^{m_1}\) and \(d \in \R^{m_2}\) are the corresponding
right\sphinxhyphen{}hand sides, \(\Delta > 0\) is the current trust\sphinxhyphen{}region radius, and
\(\norm{\cdot}\) is the Euclidean norm.


\subsection{The unconstrained case}
\label{\detokenize{algo/linalg.lctcg:the-unconstrained-case}}
\sphinxAtStartPar
As mentioned in the section {\hyperref[\detokenize{algo/linalg.bvtcg:tcg-base}]{\sphinxcrossref{\DUrole{std,std-ref}{The unconstrained case}}}} of the description of the \sphinxcode{\sphinxupquote{bvtcg}}
method, it is usual to solve inexactly problem \eqref{equation:algo/linalg.lctcg:lctcg} using a truncated
conjugate gradient method of Steihaug \sphinxcite{algo/linalg.lctcg:stei83} and Toint \sphinxcite{algo/linalg.lctcg:toin81} in the
unconstrained case. Given the initial values \(x^0 = 0\) and
\(d^0 = -g\), it generates the sequence of iterates
\begin{equation*}
\begin{split}\left\{
\begin{array}{l}
    \alpha_k = -\inner{d^k, g^k} / \inner{d^k, Hd^k},\\
    \beta_k = \norm{g^{k + 1}}^2 / \norm{g^k}^2,\\
    x^{k + 1} = x^k + \alpha_k d^k,\\
    d^{k + 1} = -g^k + \beta_k d^k,
\end{array}
\right.\end{split}
\end{equation*}
\sphinxAtStartPar
where \(g^k = \nabla f(x^k) = g + Hx^k\). The computations are stopped if
either
\begin{enumerate}
\sphinxsetlistlabels{\arabic}{enumi}{enumii}{}{.}%
\item {} 
\sphinxAtStartPar
\(g^k = 0\) and \(\inner{d^k, Hd^k} \ge 0\), in which case the
global minimizer is found; or

\item {} 
\sphinxAtStartPar
\(\norm{x^{k + 1}} \ge \Delta\) or \(\inner{d^k, Hd^k} < 0\), in
which case \(x^k + \alpha_{\Delta} d^k\) is returned, where
\(\alpha_{\Delta} > 0\) is chosen so that
\(\norm{x^k + \alpha_{\Delta} d^k} = \Delta\).

\end{enumerate}


\subsection{The constrained case}
\label{\detokenize{algo/linalg.lctcg:the-constrained-case}}
\sphinxAtStartPar
When linear and/or nonlinear constraints are provided, COBYQA solves its
trust\sphinxhyphen{}region tangential subproblems using a modified TRSTEP algorithm
\sphinxcite{algo/linalg.lctcg:powe15}. It is an active\sphinxhyphen{}set variation of the truncated conjugate gradient
algorithm, which maintains the QR factorization of the matrix whose columns are
the gradients of the active constraints. As for \sphinxcode{\sphinxupquote{bvtcg}}, if a new constraint is
added to the active set, the procedure is restarted. However, we allow
constraints to be removed from the active set in this method.


\subsubsection{Management of the active set}
\label{\detokenize{algo/linalg.lctcg:management-of-the-active-set}}
\sphinxAtStartPar
For convenience, we denote
\(a_j\), for \(j \in \set{1, 2, \dots, m_1}\) and
\(c_j\), for \(j \in \set{1, 2, \dots, m_2}\) the rows of the matrices
\(A\) and \(C\). We assume that the initial guess \(x^0\) is
feasible and that an inequality constraint \(b_j - \inner{a_j, x}\) is
positive and tiny for some \(j \le m_1\). If \(j\) do not belong to the
active set and if \(\inner{a_j, g} < 0\), then it is likely that the
\(\norm{x^1 - x^0}\) is small, as a step along the search direction
\(d^0 = -g\) quickly exits the feasible set. Therefore, we must consider a
constraint active whenever its residual becomes small. More precisely, for some
feasible \(x \in \R^n\), we let
\begin{equation*}
\begin{split}\mathcal{J}(x) = \set[\big]{j \le m_1 : b_j - \inner{a_j, x} \le \eta \Delta \norm{a_j}},\end{split}
\end{equation*}
\sphinxAtStartPar
where \(\eta\) is some positive constant
(set to \(\eta = 0.2\) in \sphinxcode{\sphinxupquote{lctcg}}). At each iteration, the active set is a
subset of \(\mathcal{J}(x^k)\). Moreover, the initial search direction
\(d^0\) should be close to \(-g\) and prevent the point \(x^1\) to
be close from \(x^0\). Therefore, the initial search direction \(d^0\)
is the unique solution of
\begin{equation*}
\begin{split}\begin{array}{ll}
    \min        & \quad \frac{1}{2} \norm{g + d}^2\\
    \text{s.t.} & \quad \inner{a_j, d} \le 0, ~ j \in \mathcal{J}(x^k),\\
                & \quad \inner{c_j, d} = 0, ~ j \in \set{1, 2, \dots, m_2}\\
                & \quad d \in \R^n.
\end{array}\end{split}
\end{equation*}
\sphinxAtStartPar
If \(\inner{a_j, d^0} < 0\) for some \(j\), then the point \(x^1\)
will be further from this constraint than the initial guess. Therefore, the
active set \(\mathcal{I}\) is chosen to be
\(\set{j \in \mathcal{J}(x^0) : \inner{a_j, d^0} = 0}\) (or a subset of it,
chosen so that \(\set{a_j : j \in \mathcal{I}}\) is a basis of
\(\vspan \set{a_j : j \in \mathcal{J}(x^0), ~ \inner{a_j, d^0} = 0}\).

\sphinxAtStartPar
The solution of such a problem is calculated using the Goldfarb and Idnani
method for quadratic programming \sphinxcite{algo/linalg.lctcg:goid83}.
\subsubsection*{References}


\section{Nonnegative least squares}
\label{\detokenize{algo/linalg.nnls:nonnegative-least-squares}}\label{\detokenize{algo/linalg.nnls:linalg-nnls}}\label{\detokenize{algo/linalg.nnls::doc}}
\sphinxAtStartPar
The update mechanism of the Lagrange multipliers in COBYQA is based on a
constrained least\sphinxhyphen{}squares problem, where some variables must remain nonnegative
(in order to satisfy some complementary slackness conditions). The problem we
solve is of the form
\begin{equation}\label{equation:algo/linalg.nnls:nnls}
\begin{split}\begin{array}{ll}
    \min        & \quad f(x) = \frac{1}{2} \norm{Ax - b}^2\\
    \text{s.t.} & \quad x_i \ge 0, ~ i = 1, 2, \dots, n_0,\\
                & \quad x \in \R^n,
\end{array}\end{split}
\end{equation}
\sphinxAtStartPar
where \(A \in \R^{m \times n}\), \(b \in \R^m\), \(n_0\) is a
nonnegative integer with \(n_0 \le n\), and \(\norm{\cdot}\) is the
Euclidean norm. We observe that if \(n_0 = 0\), problem \eqref{equation:algo/linalg.nnls:nnls} is a
simple unconstrained least\sphinxhyphen{}squares problem, which can be solved using
traditional methods (see, e.g., \sphinxhref{https://numpy.org/doc/stable/reference/generated/numpy.linalg.lstsq.html\#numpy.linalg.lstsq}{\sphinxcode{\sphinxupquote{numpy.linalg.lstsq}}}).


\subsection{Description of the method}
\label{\detokenize{algo/linalg.nnls:description-of-the-method}}
\sphinxAtStartPar
In order to solve problem \eqref{equation:algo/linalg.nnls:nnls} when \(n_0 \ge 1\), we construct an
active\sphinxhyphen{}set method based on Algorithm 23.10 of \sphinxcite{algo/linalg.nnls:laha74}, referred to as \sphinxcode{\sphinxupquote{nnls}}.
The framework of the method is described below.
\begin{enumerate}
\sphinxsetlistlabels{\arabic}{enumi}{enumii}{}{.}%
\item {} 
\sphinxAtStartPar
Set the active set \(\mathcal{I}^0\) initially to
\(\set{1, 2, \dots, n_0}\), the initial guess \(x^0\) to the
origin, and \(k = 0\).

\item {} 
\sphinxAtStartPar
Evaluate the gradient of the objective function of \eqref{equation:algo/linalg.nnls:nnls} at
\(x^k\), namely \(\nabla f(x^k) = A^{\T} (Ax^k - b)\).

\item {} 
\sphinxAtStartPar
If the KKT conditions for problem \eqref{equation:algo/linalg.nnls:nnls} hold at \(x^k\), stop the
computations.

\item {} 
\sphinxAtStartPar
Remove from the active set \(\mathcal{I}^k\) an index yielding
\(\min \set{\partial_i f(x) : i \in \mathcal{I}^k}\) to build
\(\mathcal{I}^{k + 1}\).

\item {} 
\sphinxAtStartPar
Let \(A_{\scriptscriptstyle\mathcal{I}}^{k + 1}\) be the matrix whose
\(i\)\sphinxhyphen{}th column if the \(i\)\sphinxhyphen{}th column of \(A\) if
\(i \notin \mathcal{I}^{k + 1}\), and zero otherwise.

\item {} 
\sphinxAtStartPar
Let \(z^{k + 1}\) be a solution of the least squares
\(\min \norm{A_{\scriptscriptstyle\mathcal{I}}^{k + 1} z - b}\) with
\(z_i^{k + 1} = 0\) for \(i \in \mathcal{I}^{k + 1}\), and increment
\(k\).

\item {} 
\sphinxAtStartPar
If \(z_i^k > 0\) for all \(i \notin \mathcal{I}^k\) with
\(i \le n_0\), update \(x^k = z^k\) and go to step 2. Set otherwise
\(x^k = x^{k - 1}\).

\item {} 
\sphinxAtStartPar
Set \(\alpha_k = \min \set{x_i^k / (x_i^k - z_i^k) : z_i^k \le 0, ~ i \notin \mathcal{I}^k, ~ i \le n_0}\).

\item {} 
\sphinxAtStartPar
Update \(x^{k + 1} = x^k + \alpha_k (z^k - x^k)\),
\(\mathcal{I}^{k + 1} = \mathcal{I}^k \cup \set{i \le n_0 : x_i^{k + 1} = 0}\),
increment \(k\), and go to step 5.

\end{enumerate}

\sphinxAtStartPar
The unconstrained least\sphinxhyphen{}squares subproblem of \sphinxcode{\sphinxupquote{nnls}} at step 6 is solved using
\sphinxhref{https://numpy.org/doc/stable/reference/generated/numpy.linalg.lstsq.html\#numpy.linalg.lstsq}{\sphinxcode{\sphinxupquote{numpy.linalg.lstsq}}}. Several refinements of this framework have been made in
the implementation.
\begin{enumerate}
\sphinxsetlistlabels{\arabic}{enumi}{enumii}{}{.}%
\item {} 
\sphinxAtStartPar
The number of past iterations is maintained to stop the computations if a
given threshold is exceeded.

\item {} 
\sphinxAtStartPar
Numerical difficulties may arise at step 8 of the method whenever the
denominator comes close to zero (although it remains theoretically nonzero).
The division is then safeguarded.

\item {} 
\sphinxAtStartPar
Computer rounding errors may engender infinite cycling if the solution has
been found when checking the KKT conditions. Therefore, the computations are
stopped if no progress is made from an iteration to another in terms of
objective function reduction. The theoretical analysis below show that
strict function decrease must occur from an iteration to another.

\end{enumerate}


\subsection{Convergence of the method}
\label{\detokenize{algo/linalg.nnls:convergence-of-the-method}}
\sphinxAtStartPar
To study the theoretical properties of the framework, we regard it as
consisting of two nested loops. The inner loop (steps 5–9) has unique entry
and exit points at steps 5 and 7, and only broadens the active set. The outer
loop (steps 2–9) also has unique entry and exit points at steps 2 and 3, and
only narrows the active set. Since the termination criteria of the outer loop
are the KKT conditions for problem \eqref{equation:algo/linalg.nnls:nnls}, it is clear that the termination
of the algorithm implies its convergence. In order to prove the termination of
the algorithm, we necessitate the following result whose proof is established
in Lemma 23.17 of \sphinxcite{algo/linalg.nnls:laha74}.
\begin{quote}

\sphinxAtStartPar
\sphinxstylestrong{Lemma 1.} Assume that a matrix \(A \in \R^{m \times n}\) is a full
column rank matrix (with \(n \le m\)) and that a vector
\(b \in \R^m\) satisfies \(\inner{b, Ae_i} = 0\) for
\(i \in \set{1, 2, \dots, n} \setminus \set{j}\) and
\(\inner{b, Ae_j} > 0\) with \(j \in \set{1, 2, \dots, n}\). Then
the solution vector \(x^{\ast}\) of the least\sphinxhyphen{}squares problem
\(\min \norm{Ax - b}\) satisfies \(x_j^{\ast} > 0\).
\end{quote}

\sphinxAtStartPar
Assume that the KKT conditions for problem \eqref{equation:algo/linalg.nnls:nnls} do not hold at the
origin. At the first iteration, the algorithm selects the index
(\(j\), say) of the most negative component of
\(\nabla f(0) = -A^{\T} b\) and remove it from the active set
\(\mathcal{I}^0\). According to Lemma 1, the solution of the least\sphinxhyphen{}squares
problem at step 6 satisfies \(z_j^1 > 0\). It is then easy to see that at
each iteration, the solution satisfies \(z_j^{k + 1} > 0\), where \(j\)
is the index selected at step 4. Such a solution is then modified by the inner
loop to ensure that \(x_i^{k + 1} \ge 0\) for any
\(i \in \set{1, 2, \dots, n_0}\). To do so, it will select the closest
point to \(z^{k + 1}\) on the line joining \(x^k\) to \(z^{k + 1}\)
that is feasible.


\subsubsection{Termination of the inner loop}
\label{\detokenize{algo/linalg.nnls:termination-of-the-inner-loop}}
\sphinxAtStartPar
It is clear at this step that all operations made in the inner loop are
well\sphinxhyphen{}defined. Moreover, at each inner loop iteration, the cardinal number of
the active set \(\mathcal{I}^k\) is incremented. If \(\mathcal{I}^k\)
is maximal, that is \(\mathcal{I}^k = \set{1, 2, \dots, n_0}\), then the
condition at step 7 is always true, and the loop terminates. Therefore, the
inner loop must terminate in at most \(n_0 - \abs{\mathcal{I}^k} + 1\)
iterations, where \(\abs{\mathcal{I}^k}\) denotes the cardinal number of
the active set when the first inner iteration started.


\subsubsection{Termination of the outer loop}
\label{\detokenize{algo/linalg.nnls:termination-of-the-outer-loop}}
\sphinxAtStartPar
Termination of the outer loop can be directly inferred by showing that the
value of \(f\) strictly decreases at each outer loop iteration. Such a
condition ensures that the active set \(\mathcal{I}^k\) at a given outer
loop iteration is different from all its previous instances, as the feasibility
of each iterate of the outer loop has already been shown (it is specifically
the purpose of the inner loop).

\sphinxAtStartPar
When an inner loop iteration finishes, the value of \(x^{k + 1}\) is solves
\begin{equation*}
\begin{split}\begin{array}{ll}
    \min        & \quad f(x) = \frac{1}{2} \norm{Ax - b}^2\\
    \text{s.t.} & \quad x_i = 0, ~ i \in \mathcal{I}^{k + 1},\\
                & \quad x_i > 0, ~ i \notin \mathcal{I}^{k + 1}, ~ i \le n_0,\\
                & \quad x \in \R^n.
\end{array}\end{split}
\end{equation*}
\sphinxAtStartPar
If the KKT conditions for problem \eqref{equation:algo/linalg.nnls:nnls} hold at \(x^{k + 1}\), then
termination occurs. Otherwise, after incrementing \(k\), the index yielding
the least value of \(\nabla f(x^k)\) is removed from the active set
\(\mathcal{I}^k\), and the tentative solution vector \(z^{k + 1}\)
clearly provides \(f(z^{k + 1}) < f(x^k)\). The result is then proven if no
inner loop is entertained (that is, if condition at step 7 holds). Otherwise,
at the end of each inner loop, we have
\begin{equation*}
\begin{split}\begin{aligned}
    \sqrt{2 f\big(x^{k + 1}\big)}
        &= \norm[\big]{A (x^k + \alpha_k (z^k - x^k)) - b}\\
        &= \norm[\big]{(1 - \alpha_k) (Ax^k - b) + \alpha_k (Az^k - b)}\\
        &< \norm{Ax^k - b} = \sqrt{2 f(x^k)},
\end{aligned}\end{split}
\end{equation*}
\sphinxAtStartPar
since \(\alpha_k \in (0, 1)\). The termination of the outer loop is then
proven, as well as the convergence of the method.
\subsubsection*{References}

\begin{sphinxthebibliography}{Powe75}
\bibitem[Powe75]{algo/linalg.bvtcg:powe75}
\sphinxAtStartPar
M. J. D. Powell. “Convergence properties of a class of minimization
algorithms.” In: Nonlinear Programming. Ed. by O. L. Mangasarian, R. R.
Meyer, and S. M. Robinson. Vol. 2. London, UK: Academic Press, 1975, pp.
1–27.
\bibitem[Powe09]{algo/linalg.bvtcg:powe09}
\sphinxAtStartPar
M. J. D. Powell. The BOBYQA algorithm for bound constrained
optimization without derivatives. Tech. rep. DAMTP 2009/NA06. Cambridge, UK:
Department of Applied Mathematics and Theoretical Physics, University of
Cambridge, 2009.
\bibitem[Stei83]{algo/linalg.bvtcg:stei83}
\sphinxAtStartPar
T. Steihaug. “The conjugate gradient method and trust regions in
large scale optimization.” In: IMA J. Numer. Anal. 20 (1983), pp. 626–637.
\bibitem[Toin81]{algo/linalg.bvtcg:toin81}
\sphinxAtStartPar
Ph. L. Toint. “Towards an efficient sparsity exploiting Newton
method for minimization.” In: Sparse Matrices and Their Uses. Ed. by I. S.
Duff. London, UK: Academic Press, 1981, pp. 57–88.
\bibitem[GoId83]{algo/linalg.lctcg:goid83}
\sphinxAtStartPar
D. Goldfarb and A. Idnani. “A numerically stable dual method for
solving strictly convex quadratic programs.” In: Math. Program. 27 (1983),
pp. 1–33.
\bibitem[Powe15]{algo/linalg.lctcg:powe15}
\sphinxAtStartPar
M. J. D. Powell. “On fast trust region methods for quadratic models
with linear constraints.” In: Math. Program. Comput. 7 (2015), pp. 237–267.
\bibitem[Stei83]{algo/linalg.lctcg:stei83}
\sphinxAtStartPar
T. Steihaug. “The conjugate gradient method and trust regions in
large scale optimization.” In: IMA J. Numer. Anal. 20 (1983), pp. 626–637.
\bibitem[Toin81]{algo/linalg.lctcg:toin81}
\sphinxAtStartPar
Ph. L. Toint. “Towards an efficient sparsity exploiting Newton
method for minimization.” In: Sparse Matrices and Their Uses. Ed. by I. S.
Duff. London, UK: Academic Press, 1981, pp. 57–88.
\bibitem[LaHa74]{algo/linalg.nnls:laha74}
\sphinxAtStartPar
C. L. Lawson and R. J. Hanson. Solving Least Squares Problems.
Classics Appl. Math. Philadelphia, PA, US: SIAM, 1974.
\end{sphinxthebibliography}



\renewcommand{\indexname}{Index}
\printindex
\end{document}